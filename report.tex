%!Mode:: "TeX:UTF-8"
% 模板名称:BUAA_report
% 模板版本:V0.2(2020.10.16)
% 请使用XeLaTeX进行编译
% 如果要使用超链接\ref的话,编译两次,两次,两次!!!
%
% 此模板为北京航空航天大学网络空间安全学院实验报告模板(自用)
% 基于电子科技大学信软学院实验报告模板制作(基于UESTC_Lab_Report_LaTeX_Template模板的改进版)
% 在此向这个模板的原作者和所有改进者致谢!
% 主要改动有:
%           1.将配置单独放在了settings.tex文件中。
%           2.超链接跳转功能,方便图片与文字相隔很远时进行阅读。(默认为蓝色,
%           如需打印可在settings.tex中改为黑色或不使用超链接)


\documentclass[a4paper,11pt,UTF8,AutoFakeBold= {2.88}]{ctexart}
\input{settings.tex}
% 导入配置文件settings.tex,
% 配置参数均存储在settings.tex文件中,
% 添加或修改均需在该文件中进行

\newcommand{\docname}{基于红茶的全同态加密研究报告} %在这里更改实验报告的名称

% 页眉页脚
\pagestyle{fancy}
\chead{\docname} %页眉正中
\lhead{} %左上角
\rhead{} %右上角
\cfoot{\thepage} %显示页码
\renewcommand{\headrulewidth}{0.4pt} %分割线宽度
\renewcommand{\footrulewidth}{0.4pt}

\begin{document}
\xiaosihao\song

\begin{titlepage}
\begin{figure}[!htbp]
\centering
%\includegraphics[scale=0.1]{logo}
\includegraphics[width=\textwidth]{bhlogo}
\label{bhlogo}
\end{figure}
\vspace{4cm}
\center{\shiyanbaogao{\ktgb{\textbf{\docname}}}}
\vspace{4.5cm}

\begin{center}
\begin{large}
\begin{tabular}{rc}

\xiaoerhao{\ktgb{\textbf{(实验)课程名称:}}}& \xiaoerhao{\ktgb{\textbf{下北沢几何与同态密码学}}}\\
\cline{2-2}\\

\end{tabular}
\end{large}
\end{center}

\vspace{3.5cm}
\begin{center}
  \sihao{\song{\textbf{李田所-18114514}}}
\end{center}

\end{titlepage}
\clearpage

\tableofcontents
\clearpage

\setlength{\parskip}{6pt}  %定义段间距

\section{简述}

先辈雷普未半而中道一转,今北沢二分,下北草生,此诚Coat存亡之秋也。然木毛之臣堆雪于内,小鬼之士脱粪于外者,盖追先辈之面接,欲报之于逸帝也。诚宜袒露林檎,以光先辈遗德,恢弘人中之气,不宜妄自便乘,引睿失逸,以塞忠鉴之路也。

\section{原理}

部中邸中,俱为一体,品茶雷普,不宜异同。若有求学路上为朴秀者,宜付叔叔论其刑赏,以昭逸帝人间之理,不宜营业,使小鬼恰饭也。

智将三浦,此为良实,只懂便乘,是以木村窥拔以遗逸帝。愚以为部中之事,事无大小,悉以咨之,然后便乘,必能大彻大悟,救世济民。

将军谷冈,性行淑均,晓畅车事,试驾于昔日,先辈称之曰鉴,是以众议举冈为极道。愚以为车中之事,悉以咨之,必能使狗叫三回,弹无虚发。

亲鉴臣,远屑人,此木毛所以兴隆也;亲屑人,远鉴臣,此Nonke所以倾颓也。先辈在时,每与臣面接此事,未尝不叹息周董不够快也。智将、极道、厨师、叔叔,此悉木毛股肛之臣,愿逸帝亲之信之,则逸站之隆,可计日而待也。

\section{具体结构}

\subsection{复制}

按下ctrl+c

\subsection{粘贴}

按下ctrl+v

\section{测试分析}

臣本小鬼,目力于家,便乘社畜于好时代,不求鉴达于臭乎。先辈不以臣卑鄙,亲自上课,三被臣品鉴于本篇之中,教臣以木毛之事,由是感激,遂许先辈以一个。后值倾覆,受任于dssq之际,奉命于永封之间,尔来二十有四年矣!(大嘘

先辈知臣事屑,故临脱寄臣以红茶也。受茶以来,夙夜昏睡,恐用药不效,以伤先辈之明,故九月拨雪,深入木毛。今大势已趋,小鬼已足,当雷普三军,北定陈睿,庶竭驽钝,攘除Nonke,兴复Coat,还于逸站。此臣所以报先辈而忠逸帝之职分也。至于斟酌屑鉴,进尽米青,则三浦、木村之任也。

\section{改进意见}

愿逸帝托臣以讨睿幸终之效,不效,则掘臣之腚,以告先辈之灵。若无池沼之言,则责三浦之慢,以嗦其牛。逸帝亦宜自谋,以保养林檎,茶纳雅盐,深追先辈遗沼。臣不胜脱雪感激!

\section{余论}

今当投稿,临雪涕零,不知所掘。

%\vspace{4cm}
\clearpage

\begin{appendix}

\section{代码示例}

%listings官方文档:http://texdoc.net/texmf-dist/doc/latex/listings/listings.pdf


\begin{lstlisting}[caption={一段python代码},captionpos=b,language=python]
  #!/usr/bin/python
  # -*- coding: UTF-8 -*-
   
  print("hello world")
\end{lstlisting}

支持的语言如图\ref{language}

\begin{figure}[!htbp]
  \centering
  \includegraphics[scale=0.7]{language}
  %\includegraphics[width=\textwidth]{language}
  \bottomcaption{\wuhao{language}}
  \label{language}
  \end{figure}

\section{图片示例}
超链接:运行结果如图\ref{logo}
  \begin{figure}[!htbp]
  \centering
  %\includegraphics[scale=0.1]{logo}
  \includegraphics[width=\textwidth]{logo}
  \bottomcaption{\wuhao{北京航空航天大学}}
  \label{logo}
  \end{figure}
  
\clearpage

\section{伪代码示例}

\begin{algorithm}[H]
\caption{某个算法}
\begin{algorithmic}[1]  %每行显示行号
\Require 某个输入
\Ensure 某个输出
\Function {函数名} {参数列表}
    \State 某个变量  $\gets$ 某个变量
\EndFunction
\end{algorithmic}
\end{algorithm}

\clearpage

\section{字体示例}
\hei{黑体}
\hwxk{华文行楷}

\clearpage

\section{表格示例}

\begin{table}[!h!tbp]
\caption{一个简单的表格}\label{tab1}
  \centering
  \begin{tabular}{|l|c|c|}
	\hline
	功能          &WEB         &APP         \\ \hline
	注册          &$\surd$     &$\surd$     \\ \hline
	登录          &$\surd$     &$\surd$     \\ \hline
	推送          &$\times$    &$\surd$     \\ \hline
\end{tabular}
\end{table}

\begin{table}[!h!tbp]
\caption{自定义表格}\label{tab2}
  \centering
\begin{tabular*}{0.75\textwidth}{@{\extracolsep{\fill}}lcc}
    \toprule
    功能          &WEB         &APP         \\
    \midrule
    注册          &$\surd$     &$\surd$     \\
    登录          &$\surd$     &$\surd$     \\
    推送          &$\times$    &$\surd$     \\
    \bottomrule
\end{tabular*}
\end{table}


\end{appendix}

\end{document}
