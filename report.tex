%!Mode:: "TeX:UTF-8"
% 模板名称:BUAA_report
% 模板版本:V.1(2020.10.15)
% 请使用XeLaTeX进行编译
% 如果要使用超链接\ref的话,编译两次,两次,两次!!!
%
% 此模板为北京航空航天大学网络空间安全学院实验报告模板(自用)
% 基于电子科技大学信软学院实验报告模板制作(基于UESTC_Lab_Report_LaTeX_Template模板的改进版)
% 在此向这个模板的原作者和所有改进者致谢!
% 主要改动有:
%           1.将配置单独放在了settings.tex文件中。
%           2.超链接跳转功能,方便图片与文字相隔很远时进行阅读。(默认为蓝色,
%           如需打印可在settings.tex中改为黑色或不使用超链接)


\documentclass[a4paper,11pt,UTF8,AutoFakeBold= {2.88}]{ctexart}
\input{settings.tex}
% 导入配置文件settings.tex,
% 配置参数均存储在settings.tex文件中,
% 添加或修改均需在该文件中进行

% 页眉页脚
\pagestyle{fancy}
\chead{课程名称:实验报告标题} %页眉正中
\rhead{} %右上角
\cfoot{\thepage} %显示页码
\renewcommand{\headrulewidth}{0.4pt} %分割线宽度
\renewcommand{\footrulewidth}{0.4pt}

\begin{document}
\xiaosihao\song

\begin{titlepage}
  \leftline{\ktgb{\textbf{\yihao{ 北京航空航天大学}\underline{\xiaoyihao{网络空间安全学院}}}}}
\vspace{5.5cm}
\center{\shiyanbaogao{\ktgb{\textbf{标~ 准~ 实~ 验~ 报~ 告}}}}
\vspace{5.5cm}

\begin{center}
\begin{large}
\begin{tabular}{rc}

\xiaoerhao{\ktgb{\textbf{(实验)课程名称:}}}& \xiaoerhao{\ktgb{\textbf{XXXX}}}\\
\cline{2-2}\\

\end{tabular}
\end{large}
\end{center}

\vspace{5cm}
\begin{center}
  \sihao{\song{\textbf{李田所-18114514}}}
\end{center}

\end{titlepage}
\clearpage


\leftline{\\[10pt]\sihao{\textbf{\song{姓名:XXX \hfill 学号:XXXX \hfill 实验时间:XXXX-XX-XX }}}}




\setlength{\parskip}{6pt}  %定义段间距

\section{实验项目名称:XXX}



\section{实验原理:}



\section{实验步骤:}

\subsection{复制}

按下ctrl+c

\subsection{粘贴}

按下ctrl+v

\section{结果分析:}


\section{实验结论、心得体会:}


\section{改进建议:}



%\vspace{4cm}
\clearpage

\begin{appendix}

\section{代码示例}

%listings官方文档:http://texdoc.net/texmf-dist/doc/latex/listings/listings.pdf


\begin{lstlisting}[caption={一段python代码},captionpos=b,language=python]
  #!/usr/bin/python
  # -*- coding: UTF-8 -*-
   
  print("hello world")
\end{lstlisting}

支持的语言如图\ref{language}

\begin{figure}[!htbp]
  \centering
  \includegraphics[scale=0.7]{language}
  %\includegraphics[width=\textwidth]{language}
  \bottomcaption{\wuhao{language}}
  \label{language}
  \end{figure}

\section{图片示例}
超链接:运行结果如图\ref{logo}
  \begin{figure}[!htbp]
  \centering
  %\includegraphics[scale=0.1]{logo}
  \includegraphics[width=\textwidth]{logo}
  \bottomcaption{\wuhao{北京航空航天大学}}
  \label{logo}
  \end{figure}
  
\clearpage

\section{伪代码示例}

\begin{algorithm}
\caption{某个算法}
\begin{algorithmic}[1]  %每行显示行号
\Require 某个输入
\Ensure 某个输出
\Function {函数名} {参数列表}
    \State 某个变量  $\gets$ 某个变量
\EndFunction
\end{algorithmic}
\end{algorithm}

\clearpage

\section{字体示例}
\hei{黑体}
\hwxk{华文行楷}

\clearpage

\section{表格示例}

\begin{table}[!h!tbp]
\caption{一个简单的表格}\label{tab1}
  \centering
  \begin{tabular}{|l|c|c|}
	\hline
	功能          &WEB         &APP         \\ \hline
	注册          &$\surd$     &$\surd$     \\ \hline
	登录          &$\surd$     &$\surd$     \\ \hline
	推送          &$\times$    &$\surd$     \\ \hline
\end{tabular}
\end{table}

\begin{table}[!h!tbp]
\caption{自定义表格}\label{tab2}
  \centering
\begin{tabular*}{0.75\textwidth}{@{\extracolsep{\fill}}lcc}
    \toprule
    功能          &WEB         &APP         \\
    \midrule
    注册          &$\surd$     &$\surd$     \\
    登录          &$\surd$     &$\surd$     \\
    推送          &$\times$    &$\surd$     \\
    \bottomrule
\end{tabular*}
\end{table}


\end{appendix}

\end{document}
